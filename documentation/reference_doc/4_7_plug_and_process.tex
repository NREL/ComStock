\section{Plug and Process Loads}
\label{sec:plug_and_process_loads}

Plug and process loads (PPLs) are all electrical or gas building loads that do not fall under lighting, heating, cooling, ventilation, or water heating. As lighting and HVAC equipment becomes more efficient, PPLs represent an increasing percentage of commercial building energy consumption---up to 50\% in high-performance buildings. This section describes how electric equipment, gas equipment, data centers, elevators, and kitchen equipment are modeled in ComStock. 

\subsection{Electric Equipment}

Electric equipment is a broad category that encompasses any type of load that is powered by an AC outlet. This can include computers, monitors, printers, kitchen or bathroom appliances, laundry equipment, phone chargers, and more. Because there are so many types of electric equipment, ComStock does not model each technology separately, but instead uses an equipment power density (EPD) value for each space type, in watts per square foot. 

% TODO check if we can actually name Enertiv and Powerhouse Dynamics
The EPDs are derived from the DOE prototype buildings; however, some of the values were adjusted during the end-use load profiles calibration process. Using end-use-level data provided by two industry sources for a variety of building types, we increased or decreased some EPD assumptions to better reflect the actual building data. Building types that were affected by the EPD adjustments included Full Services Restaurant (FullServiceRestaurant), Primary School (PrimarySchool), Quick Service Restaurant (QuickServiceRestaurant), Retail (Retail), Secondary School (SecondarySchool), and Strip Mall (StripMall). 

The EPDs are dependent on building type, space type, and DOE Reference Building template. The interior equipment template is a function of the vintage of the building, as well as equipment turnover assumptions. In most cases, the EPD remains constant for all templates; however, some values increase or decrease in newer templates. An increase in the EPD in newer templates for a particular space type most likely indicates that new types of plug loads or technologies are assumed to be in the space. By comparison, a decrease in EPD indicates that plug loads in that space have become more efficient as buildings upgrade to newer equipment. For example, EPDs in the ``MediumOffice - Conference'' space type decrease beginning with the 90.1-2004 template, reflecting an assumption that conference equipment such as projectors and monitors have become more efficient. On the other hand, EPDs in ``MediumOffice - Breakroom'' increase drastically, likely due to the addition of new kitchen appliances and accessories. 

The EPDs in the ComStock model for each combination of building type, space type, and template are shown in Table ~\ref{tab:epds}.

%\begin{center}
\scriptsize
\begin{longtable}{p{1.25in}p{1in}p{0.5in}p{0.5in}p{0.5in}p{0.5in}p{0.5in}p{0.5in}}
\caption[Electric Equipment Power Density]{Electric Equipment Power Density by Space Type} \\ \hline
\label{tab:epds}
\textbf{Building Type} & \textbf{Space Type} & \textbf{Pre-1980} & \textbf{1980-2004} & \textbf{90.1-2004} & \textbf{90.1-2007} & \textbf{90.1-2010} & \textbf{90.1-2013} \\ \hline
\endfirsthead
\multicolumn{2}{c} {\textbf{\textit{Continued from previous page}}} \\ \hline
\textbf{Building Type} & \textbf{Space Type} & \textbf{Pre-1980} & \textbf{1980--2004} & \textbf{90.1-2004} & \textbf{90.1-2007} & \textbf{90.1-2010} & \textbf{90.1-2013} \\ \hline
\endhead
FullServiceRestaurant              & Dining                                & 2.5     & 2.5     & 2.5    & 2.5   & 2.5   & 2.5    \\
                                   & Basement                             & 0.7     & 0.7     & 0.8    & 0.8   & 0.8   & 0.8    \\
                                   & Corridor                             & 0.0     & 0.0     & 0.0    & 0.0   & 0.0   & 0.0    \\
                                   & Dining                               & 1.0     & 1.0     & 1.0    & 1.0   & 1.0   & 1.0    \\
                                   & ER\_Exam                             & 1.5     & 1.5     & 1.5    & 1.5   & 1.5   & 1.5    \\
                                   & ER\_NurseStn                         & 1.4     & 1.4     & 1.4    & 1.4   & 1.4   & 1.4    \\
                                   & ER\_Trauma                           & 4.0     & 4.0     & 4.0    & 4.0   & 4.0   & 4.0    \\
                                   & ER\_Triage                           & 2.7     & 2.7     & 2.0    & 2.0   & 2.0   & 2.0    \\
                                   & HospitalOfficeFlr1                   &         &         &        &       &       & 1.1    \\
                                   & HospitalOfficeFlr5                   &         &         &        &       &       & 1.0    \\
                                   & ICU\_NurseStn                        & 2.0     & 2.0     & 2.0    & 2.0   & 2.0   & 2.0    \\
                                   & ICU\_Open                            & 3.0     & 3.0     & 3.0    & 3.0   & 3.0   & 3.0    \\
                                   & ICU\_PatRm                           & 3.0     & 3.0     & 3.0    & 3.0   & 3.0   & 3.0    \\
                                   & Lab                                  & 4.0     & 4.0     & 4.0    & 4.0   & 4.0   & 4.0    \\
                                   & Lobby                                & 0.1     & 0.1     & 0.1    & 0.1   & 0.1   & 0.1    \\
                                   & NurseStn                             & 1.0     & 1.0     & 1.0    & 1.0   & 1.0   & 1.0    \\
                                   & Office                               & 1.0     & 1.0     & 1.0    & 1.0   & 1.0   & 1.0    \\
                                   & OR                                   & 4.0     & 4.0     & 4.0    & 4.0   & 4.0   & 4.0    \\
                                   & PatCorridor                          & 0.0     & 0.0     & 0.0    & 0.0   & 0.0   & 0.0    \\
                                   & PatRoom                              & 2.0     & 2.0     & 2.0    & 2.0   & 2.0   & 2.0    \\
                                   & PhysTherapy                          & 1.5     & 1.5     & 1.5    & 1.5   & 1.5   & 1.5    \\
Hospital                           & Radiology                            & 4.9     & 4.9     & 10.0   & 10.0  & 10.0  & 10.0   \\
                                   & Banquet                              & 6.3     & 6.3     & 6.3    & 6.3   & 6.3   & 6.3    \\
                                   & Basement                             & 0.5     & 0.5     & 0.5    & 0.5   & 0.5   & 0.5    \\
                                   & Cafe                                 & 0.5     & 0.5     & 0.5    & 0.5   & 0.5   & 0.5    \\
                                   & Corridor                             & 0.0     & 0.0     & 0.0    & 0.0   & 0.0   & 0.0    \\
                                   & GuestRoom                            & 1.3     & 1.3     & 0.6    & 0.6   & 0.6   & 0.6    \\
                                   & GuestRoom2                           & 1.3     & 1.3     & 1.0    & 1.0   & 1.0   & 1.0    \\
                                   & GuestRoom3                           & 1.3     & 1.3     & 0.6    & 0.6   & 0.6   & 0.6    \\
                                   & GuestRoom4                           & 1.3     & 1.3     & 1.0    & 1.0   & 1.0   & 1.0    \\
                                   & GuestRoom8                           &         &         & 1.0    & 1.0   & 1.0   & 1.0    \\
                                   & Laundry                              & 5.7     & 5.7     & 5.7    & 5.7   & 5.7   & 5.7    \\
                                   & Lobby                                & 0.8     & 0.8     & 0.8    & 0.8   & 0.8   & 0.8    \\
                                   & Mechanical                           & 0.5     & 0.5     & 0.5    & 0.5   & 0.5   & 0.5    \\
                                   & Retail                               & 1.0     & 1.0     & 1.0    & 1.0   & 1.0   & 1.0    \\
                                   & Retail2                              & 1.0     & 1.0     & 1.0    & 1.0   & 1.0   & 1.0    \\
LargeHotel                         & Storage                              & 0.3     & 0.3     & 0.3    & 0.3   & 0.3   & 0.3    \\
                                   & BreakRoom                            & 5.6     & 5.6     & 5.6    & 4.5   & 4.5   & 4.5    \\
                                   & Classroom                            & 0.9     & 0.9     & 0.9    & 0.9   & 0.9   & 0.9    \\
                                   & ClosedOffice                         & 0.9     & 0.9     & 0.9    & 0.6   & 0.6   & 0.6    \\
                                   & Conference                           & 1.0     & 1.0     & 1.0    & 0.4   & 0.4   & 0.4    \\
                                   & Corridor                             & 0.3     & 0.3     & 0.3    & 0.2   & 0.2   & 0.2    \\
                                   & Dining                               & 1.0     & 1.0     & 1.0    & 1.0   & 1.0   & 1.0    \\
                                   & Elec/MechRoom                        & 0.3     & 0.3     & 0.3    & 0.3   & 0.3   & 0.3    \\
                                   & IT\_Room                             & 2.0     & 2.0     & 2.0    & 1.6   & 1.6   & 1.6    \\
                                   & Lobby                                & 0.3     & 0.3     & 0.3    & 0.1   & 0.1   & 0.1    \\
                                   & MediumOffice - Breakroom             & 1.3     & 1.3     & 5.6    & 5.6   & 5.6   & 5.6    \\
                                   & MediumOffice - Classroom             & 1.2     & 1.2     & 0.9    & 0.9   & 0.9   & 0.9    \\
                                   & MediumOffice - ClosedOffice          & 1.2     & 1.2     & 0.9    & 0.9   & 0.9   & 0.9    \\
                                   & MediumOffice - Conference            & 1.3     & 1.3     & 1.0    & 1.0   & 1.0   & 1.0    \\
                                   & MediumOffice - Corridor              & 0.4     & 0.4     & 0.3    & 0.3   & 0.3   & 0.3    \\
                                   & MediumOffice - Dining                & 1.3     & 1.3     & 1.0    & 1.0   & 1.0   & 1.0    \\
                                   & MediumOffice - Elec/MechRoom         & 0.4     & 0.4     & 0.3    & 0.3   & 0.3   & 0.3    \\
                                   & MediumOffice - Lobby                 & 0.4     & 0.4     & 0.3    & 0.3   & 0.3   & 0.3    \\
                                   & MediumOffice - OpenOffice            & 1.3     & 1.3     & 1.0    & 1.0   & 1.0   & 1.0    \\
                                   & MediumOffice - Restroom              & 0.4     & 0.4     & 0.3    & 0.3   & 0.3   & 0.3    \\
                                   & MediumOffice - Stair                 & 0.0     & 0.0     & 0.0    & 0.0   & 0     & 0.0    \\
                                   & MediumOffice - Storage               & 0.0     & 0.0     & 0.0    & 0.0   & 0.0   & 0.0    \\
                                   & OfficeLarge Data Center              & 20.0    & 20.0    & 20.0   & 20.0  & 20.0  & 20.0   \\
                                   & OfficeLarge Main Data Center         & 45.0    & 45.0    & 45.0   & 45.0  & 45.0  & 45.0   \\
                                   & OpenOffice                           & 1.0     & 1.0     & 1.0    & 0.7   & 0.7   & 0.7    \\
                                   & PrintRoom                            & 5.4     & 5.4     & 5.4    & 2.8   & 2.8   & 2.8    \\
                                   & Restroom                             & 0.3     & 0.3     & 0.3    & 0.1   & 0.1   & 0.1    \\
                                   & SmallOffice - Breakroom              & 1.3     & 1.3     & 5.6    & 5.6   & 5.6   & 5.6    \\
                                   & SmallOffice - Classroom              & 1.2     & 1.2     & 0.9    & 0.9   & 0.9   & 0.9    \\
                                   & SmallOffice - ClosedOffice           & 1.2     & 1.2     & 0.9    & 0.9   & 0.9   & 0.9    \\
                                   & SmallOffice - Conference             & 1.3     & 1.3     & 1.0    & 1.0   & 1.0   & 1.0    \\
                                   & SmallOffice - Corridor               & 0.4     & 0.4     & 0.3    & 0.3   & 0.3   & 0.3    \\
                                   & SmallOffice - Dining                 & 1.3     & 1.3     & 1.0    & 1.0   & 1.0   & 1.0    \\
                                   & SmallOffice - Elec/MechRoom          & 0.4     & 0.4     & 0.3    & 0.3   & 0.3   & 0.3    \\
                                   & SmallOffice - Lobby                  & 0.4     & 0.4     & 0.3    & 0.3   & 0.3   & 0.3    \\
                                   & SmallOffice - OpenOffice             & 1.3     & 1.3     & 1.0    & 1.0   & 1.0   & 1.0    \\
                                   & SmallOffice - Restroom               & 0.4     & 0.4     & 0.3    & 0.3   & 0.3   & 0.3    \\
                                   & SmallOffice - Stair                  & 0.0     & 0.0     & 0.0    & 0.0   & 0.0   & 0.0    \\
                                   & SmallOffice - Storage                & 0.0     & 0.0     & 0.0    & 0.0   & 0.0   & 0.0    \\
                                   & Stair                                & 0.0     & 0.0     & 0.0    & 0.0   & 0.0   & 0.0    \\
                                   & Storage                              & 0.0     & 0.0     & 0.0    & 0.0   & 0.0   & 0.0    \\
                                   & Vending                              & 3.9     & 3.9     & 3.9    & 3.9   & 3.9   & 3.9    \\
                                   & WholeBuilding - Lg Office            & 1.0     & 1.0     & 0.8    & 0.8   & 0.8   & 0.8    \\
                                   & WholeBuilding - Lg   Office-basement &         &         & 0.5    & 0.5   & 0.5   & 0.5    \\
                                   & WholeBuilding - Lg   Office-others   &         &         & 0.8    & 0.8   & 0.8   & 0.8    \\
                                   & WholeBuilding - Md Office            & 1.0     & 1.0     & 0.8    & 0.8   & 0.8   & 0.8    \\
Office                             & WholeBuilding - Sm Office            & 1.0     & 1.0     & 0.6    & 0.6   & 0.6   & 0.6    \\
                                   & Anesthesia                           & 2.0     & 2.0     & 2.0    & 2.0   & 2.0   & 2.0    \\
                                   & BioHazard                            & 0.1     & 0.1     & 0.1    & 0.1   & 0.1   & 0.1    \\
                                   & Cafe                                 & 1.0     & 1.0     & 1.0    & 1.0   & 1.0   & 1.0    \\
                                   & CleanWork                            & 2.0     & 2.0     & 2.0    & 2.0   & 2.0   & 2.0    \\
                                   & Conference                           & 1.0     & 1.0     & 1.0    & 1.0   & 1.0   & 1.0    \\
                                   & DressingRoom                         & 1.1     & 1.1     & 1.1    & 1.1   & 1.1   & 1.1    \\
                                   & Elec/MechRoom                        & 5.0     & 5.0     & 5.0    & 5.0   & 5.0   & 5.0    \\
                                   & ElevatorPumpRoom                     & 0.0     & 0.0     & 5.3    & 5.3   & 3.5   & 2.1    \\
                                   & Exam                                 & 1.1     & 1.1     & 1.1    & 1.1   & 1.1   & 1.1    \\
                                   & Hall                                 & 0.4     & 0.4     & 0.4    & 0.4   & 0.4   & 0.4    \\
                                   & Hall\_infil                          & 0.4     & 0.4     & 0.4    & 0.4   & 0.4   & 0.4    \\
                                   & IT\_Room                             & 1.1     & 1.1     & 1.1    & 1.1   & 1.1   & 1.1    \\
                                   & Janitor                              & 0.0     & 0.0     & 0.1    & 0.1   & 0.1   & 0.1    \\
                                   & Lobby                                & 1.1     & 1.1     & 1.1    & 1.1   & 1.1   & 1.1    \\
                                   & LockerRoom                           & 3.0     & 3.0     & 3.0    & 3.0   & 3.0   & 3.0    \\
                                   & Lounge                               & 3.0     & 3.0     & 3.0    & 3.0   & 3.0   & 3.0    \\
                                   & MedGas                               & 0.1     & 0.1     & 0.1    & 0.1   & 0.1   & 0.1    \\
                                   & MRI                                  & 53.3    & 53.3    & 10.0   & 10.0  & 10.0  & 10.0   \\
                                   & MRI\_Control                         & 1.1     & 1.1     & 10.0   & 10.0  & 10.0  & 10.0   \\
                                   & NurseStation                         & 2.0     & 2.0     & 2.0    & 2.0   & 2.0   & 2.0    \\
                                   & Office                               & 1.1     & 1.1     & 1.1    & 1.1   & 1.1   & 1.1    \\
                                   & OR                                   & 4.0     & 4.0     & 4.0    & 4.0   & 4.0   & 4.0    \\
                                   & OutpatientFloor2Work                 &         &         &        &       &       & 0.9    \\
                                   & PACU                                 & 3.0     & 3.0     & 3.0    & 3.0   & 3.0   & 3.0    \\
                                   & PhysicalTherapy                      & 1.5     & 1.5     & 1.5    & 1.5   & 1.5   & 1.5    \\
                                   & PreOp                                & 2.0     & 2.0     & 2.0    & 2.0   & 2.0   & 2.0    \\
                                   & ProcedureRoom                        & 3.0     & 3.0     & 3.0    & 3.0   & 3.0   & 3.0    \\
                                   & Reception                            & 1.1     & 1.1     & 1.1    & 1.1   & 1.1   & 1.1    \\
                                   & Soil Work                            & 2.0     & 2.0     & 2.0    & 2.0   & 2.0   & 2.0    \\
                                   & Stair                                & 0.0     & 0.0     & 0.4    & 0.4   & 0.4   & 0.4    \\
                                   & Toilet                               & 0.7     & 0.7     & 0.4    & 0.4   & 0.4   & 0.4    \\
                                   & Undeveloped                          & 0.0     & 0.0     & 1.1    & 1.1   & 1.1   & 1.1    \\
Outpatient                         & Xray                                 & 1.3     & 1.3     & 10.0   & 10.0  & 10.0  & 10.0   \\
                                   & Cafeteria                            & 0.3     & 0.3     & 0.3    & 0.3   & 0.3   & 0.3    \\
                                   & Classroom                            & 0.2     & 0.2     & 0.2    & 0.2   & 0.2   & 0.2    \\
                                   & ComputerRoom                         &         &         & 0.2    & 0.2   & 0.2   & 0.2    \\
                                   & Corridor                             & 0.0     & 0.0     & 0.0    & 0.0   & 0.0   & 0.0    \\
                                   & Gym                                  & 0.1     & 0.1     & 0.1    & 0.1   & 0.1   & 0.1    \\
                                   & Library                              & 0.2     & 0.2     & 0.2    & 0.2   & 0.2   & 0.2    \\
                                   & Lobby                                & 0.0     & 0.0     & 0.0    & 0.0   & 0.0   & 0.0    \\
                                   & Mechanical                           & 0.0     & 0.0     & 0.1    & 0.1   & 0.1   & 0.1    \\
                                   & Office                               & 0.1     & 0.1     & 0.1    & 0.1   & 0.1   & 0.1    \\
PrimarySchool                      & Restroom                             & 0.0     & 0.0     & 0.0    & 0.0   & 0.0   & 0.0    \\
QuickServiceRestaurant             & Dining                               & 5.5     & 5.5     & 5.6    & 5.6   & 5.6   & 5.6    \\
                                   & Back\_Space                          & 0.6     & 0.6     & 0.6    & 0.6   & 0.6   & 0.6    \\
                                   & Core\_Retail                         &         &         &        &       & 0.2   & 0.2    \\
                                   & Entry                                & 0.2     & 0.2     & 0.2    & 0.2   & 0.2   & 0.2    \\
                                   & Front\_Retail                        &         &         &        &       & 0.2   & 0.2    \\
                                   & Point\_of\_Sale                      & 1.5     & 1.5     & 1.5    & 1.5   & 1.5   & 1.5    \\
Retail                             & Retail                               & 0.2     & 0.2     & 0.2    & 0.2   & 0.2   & 0.2    \\
                                   & Auditorium                           & 0.4     & 0.4     & 0.1    & 0.1   & 0.1   & 0.1    \\
                                   & Cafeteria                            & 0.1     & 0.1     & 0.2    & 0.2   & 0.2   & 0.2    \\
                                   & Classroom                            & 0.2     & 0.2     & 0.3    & 0.3   & 0.3   & 0.3    \\
                                   & ComputerRoom                         &         &         & 0.1    & 0.1   & 0.1   & 0.1    \\
                                   & Corridor                             & 0.1     & 0.1     & 0.1    & 0.1   & 0.1   & 0.1    \\
                                   & Gym                                  & 0.2     & 0.2     & 0.2    & 0.2   & 0.2   & 0.2    \\
                                   & Gym - audience                       & 0.1     & 0.1     &        &       &       &        \\
                                   & Library                              & 0.1     & 0.1     & 0.1    & 0.1   & 0.1   & 0.1    \\
                                   & Lobby                                & 0.1     & 0.1     & 0.1    & 0.1   & 0.1   & 0.1    \\
                                   & Mechanical                           & 0.1     & 0.1     & 0.2    & 0.2   & 0.2   & 0.2    \\
                                   & Office                               & 0.2     & 0.2     & 0.1    & 0.1   & 0.1   & 0.1    \\
SecondarySchool                    & Restroom                             & 0.1     & 0.1     & 1.2    & 1.2   & 1.2   & 1.2    \\
                                   & Corridor                             & 0.0     & 0.0     & 0.0    & 0.0   & 0.0   & 0.0    \\
                                   & Elec/MechRoom                        & 198.2   & 198.2   & 198.2  & 198.2 & 198.2 & 198.2  \\
                                   & Exercise                             & 1.1     & 1.1     & 1.7    & 1.7   & 1.7   & 1.7    \\
                                   & GuestLounge                          & 1.4     & 1.4     & 2.4    & 2.4   & 2.4   & 2.4    \\
                                   & GuestRoom                            & 1.3     & 1.3     &        &       &       & 0.1    \\
                                   & GuestRoom4Occ                        & 1.3     & 1.3     & 1.1    & 1.1   & 1.1   & 1.1    \\
                                   & GuestRoom4Vac                        & 1.3     & 1.3     & 1.1    & 1.1   & 1.1   & 1.1    \\
                                   & GuestRoom123Occ                      & 1.3     & 1.3     & 1.1    & 1.1   & 1.1   & 1.1    \\
                                   & GuestRoom123Vac                      & 1.3     & 1.3     & 1.1    & 1.1   & 1.1   & 1.1    \\
                                   & Laundry                              & 2.0     & 2.0     & 2.6    & 2.6   & 2.6   & 2.6    \\
                                   & Mechanical                           & 0.0     & 0.0     & 0.0    & 0.0   & 0.0   & 0.0    \\
                                   & Meeting                              & 1.2     & 1.2     & 0.6    & 0.6   & 0.6   & 0.6    \\
                                   & Office                               & 1.2     & 1.2     & 1.3    & 1.3   & 1.3   & 1.3    \\
                                   & PublicRestroom                       & 1.0     & 1.0     & 0.0    & 0.0   & 0.0   & 0.0    \\
                                   & StaffLounge                          & 7.2     & 7.2     & 2.0    & 2.0   & 2.0   & 2.0    \\
                                   & Stair                                & 0.0     & 0.0     & 0.0    & 0.0   & 0.0   & 0.0    \\
SmallHotel                         & Storage                              & 0.0     & 0.0     & 0.0    & 0.0   & 0.0   & 0.0    \\
                                   & Strip mall - type 1                  & 0.4     & 0.4     & 0.4    & 0.4   & 0.4   & 0.4    \\
                                   & Strip mall - type 2                  & 0.4     & 0.4     & 0.4    & 0.4   & 0.4   & 0.4    \\
StripMall                          & Strip mall - type 3                  & 0.4     & 0.4     & 0.4    & 0.4   & 0.4   & 0.4    \\
                                   & Bulk                                 & 0.8     & 0.8     & 0.8    & 0.8   & 0.8   & 0.8    \\
                                   & Fine                                 & 0.0     & 0.0     & 0.0    & 0.0   & 0.0   & 0.0    \\
Warehouse                          & Office                               & 2.6     & 2.6     & 2.6    & 2.6   & 2.6   & 2.6    \\
\end{longtable}
\end{center}

\subsection{Gas Equipment}

Gas equipment refers to any natural gas-powered interior equipment that is not used for space heating or water heating. Similar to electric equipment, there are many different types of gas equipment, so ComStock does not model each technology individually, but rather uses a gas intensity in BTU per hour per square foot. Gas kitchen equipment makes up the majority of the gas equipment modeled in ComStock. Kitchen equipment will be discussed separately in Section 4.6.5. There are only three non-kitchen space types in our models that contain non-zero gas equipment values, and the values used are shown in Table~\ref{tab:gas_equip}. 

\begin{table}
\small
\centering
\caption[Gas Equipment Power Density]{Gas Equipment Power Density (Btu/hr*ft\textsuperscript{2})}
\label{tab:gas_equip}
\begin{tabular}{|l|l|r|r|r|r|r|r|}
\hline
\multicolumn{2}{|c|}{} & \multicolumn{6}{c|}{{\textbf{Template}}} \\ \hline
\textbf{Building Type} & \textbf{Space Type} & \textbf{Pre-1980} & \textbf{1980--2004} & \textbf{90.1-2004} & \textbf{90.1-2007} & \textbf{90.1-2010} & \textbf{90.1-2013} \\ \hline
LargeHotel & Laundry  & 170.0 & 170.0 & 170.0 & 170.0 & 170.0 & 170.0 \\ \hline
Outpatient & OR      & 23.9 & 23.9 & 23.9 & 23.9 & 23.9 & 23.9 \\ \hline
SmallHotel & Laundry  & 58.4 & 58.4 & 129.9 & 129.9 & 129.9 & 129.9 \\ \hline
\end{tabular}
\end{table}

The space types that contain gas equipment are the laundry and operating room space types in hotels and outpatient buildings, respectively. Gas laundry equipment represents gas clothes dryers, which are common in commercial drying applications. In operating rooms, a small amount of gas equipment represents steam sterilizers or autoclaves, which are used for sterilization during surgical procedures.

\subsection{Data Centers}

Data centers are a high-intensity type of PPL that house IT and computing equipment. Large standalone data centers are not currently modeled in ComStock, but this building type may be added in the future. Instead, ComStock models data centers as a space type within large and medium office buildings. This is meant to represent an IT closet or high-performance computer that is located within an office building and used by a business or organization. 

The data center is divided into two space types---a core data center and an IT closet. The core data center represents about 96\% of the data center floor area and has an equipment power density of 45 W/ft\textsuperscript{2}. The IT closet represents the remaining 4\% of the data center floor area and has an equipment power density of 20 W/ft\textsuperscript{2}. The area-weighted equipment power density of the whole data center is 44 W/ft\textsuperscript{2}, which is approximately 20--50 times the equipment load of most other space types \citep{osti_1129366}.

In the DOE large office prototype model, the data center represents 2.5\% of the total floor area. The medium office prototype model does not contain a data center space type. If we used the exact space type ratios from the prototype models for ComStock, all large offices would contain a data center, but no medium or small offices would contain this space type. In reality, not all office buildings contain data centers, and they can be present in offices of different sizes. Therefore, ComStock models data centers in a portion of large and medium office buildings. To determine these distributions, we used CBECS 2012 data to understand how prevalent data centers are in office buildings of different sizes. In addition, we used CBECS responses to determine what percent of a typical office building's floor area is dedicated to the data center. From this analysis, we decided that 38\% of large offices and 20\% of medium offices should contain data centers. In buildings with data centers, that space should make up approximately 2\% of the total square footage of the building. We also determined that data centers are uncommon in small offices; therefore, there is no data center space type in the small office models. 

Data centers follow a very different schedule than the rest of a building's plug and process loads. This type of IT equipment often runs 24 hours a day; therefore, the data center space type has a constant schedule year round. The start and stop time and base-to-peak ratio (BPR) schedule adjustments do not affect the data center space type. 

\subsection{Elevators}
Elevators are a high power density equipment load present in many commercial buildings. According to the Americans with Disabilities Act (ADA), elevators are required in all commercial buildings with three or more stories, or when the square footage of each floor is more than 3,000 square feet. Although not a requirement, many two-story buildings also contain elevators for accessibility and convenience. Therefore, ComStock includes elevators in all buildings with two or more stories.

Hydraulic elevators are assumed to be installed in buildings with two to six stories, and traction elevators are assumed to be installed in buildings taller than six stories. Hydraulic elevators use a fluid-driven piston to lift the cab, and typically operate at speeds of 150 feet per minute or less. Hydraulic elevators are more affordable and can carry heavier loads, but because of their slow speeds, they are typically only used in buildings up to five stories. For buildings with six or more stories, traction elevators are used because they operate at much higher speeds (up to 500 feet per minute). Traction elevators use a counterweight and pulley system, making them more energy efficient because the motor does not have to move as much weight. The drawbacks, however, include high installation and maintenance costs and limits on cab weights. The motor power is assumed to be 16,055 W for hydraulic elevators and 20,370 W for traction elevators. 

Elevators are modeled as a zone load in EnergyPlus, meaning the elevator equipment load and associated heat gain are attributed to a thermal zone. Elevator load is reported out as part of the electric equipment end use. With hydraulic elevators, the elevator room is typically located in the basement, so the equipment load and heat gain are added to the first floor core zone. With traction elevators, the elevator equipment is located on the roof, so the equipment load and heat gain are added to the top floor core zone.

The number of elevators installed in a ComStock building depends on the building type. For most building types, the number of elevators is based on the floor area. The exceptions are hospitals and hospitality buildings, for which assumptions are based on the number of hospital beds or hotel rooms, respectively. ComStock differentiates between passenger elevators and freight elevators in order to properly capture the elevator load in certain building types with industrial or service elevators. Passenger elevators are modeled in all building types, whereas freight elevators are only modeled in hospital, large hotel, large office, and warehouse buildings. The assumptions for the number of passenger and freight elevators modeled by building type are shown in Tables ~\ref{tab:passenger_elevators} and ~\ref{tab:freight_elevators}.

%\begin{table}
\small
\centering
\caption[Passenger Elevators]{Passenger Elevators}
\label{tab:passenger_elevators}
\begin{tabular}{|p{1.5in}|p{1in}|p{1in}|p{1in}|}
\hline
\textbf{Building Type}  & \textbf{Avg. Area Per Passenger Elevator (ft\textsuperscript{2})} & \textbf{Avg. Beds Per Passenger Elevator} & \textbf{Avg. Units Per Passenger Elevator} \\ \hline
FullServiceRestaurant  & 15,000    &     &     \\ \hline
Hospital               &           & 100 &     \\ \hline
LargeHotel             &           &     & 75  \\ \hline
LargeOffice            & 45,000    &     &     \\ \hline
MediumOffice           & 45,000    &     &     \\ \hline
Outpatient             & 15,000    &     &     \\ \hline
PrimarySchool          & 100,000   &     &     \\ \hline
QuickServiceRestaurant & 15,000    &     &     \\ \hline
Retail                 & 45,000    &     &     \\ \hline
SecondarySchool        & 100,000   &     &     \\ \hline
SmallHotel             &           &     & 75  \\ \hline
SmallOffice            & 45,000    &     &     \\ \hline
StripMall              & 45,000    &     &     \\ \hline
Warehouse              & 100,000   &     &     \\ \hline
\end{tabular}
\end{table}

%\begin{table}
\small
\centering
\caption[Freight Elevators]{Freight Elevators}
\label{tab:freight_elevators}
\begin{tabular}{|p{1.5in}|p{1in}|p{1in}|p{1in}|}
\hline
\textbf{Building Type} & \textbf{Avg. Area Per Freight Elevator   (ft\textsuperscript{2})} & \textbf{Avg. Beds Per Freight Elevator} & \textbf{Avg. Units Per Freight Elevator} \\ \hline
{Hospital}        &          & 100       &      \\ \hline
{LargeHotel}      &          &           & 150  \\ \hline
{LargeOffice}     & 500,000  &           &      \\ \hline
{Warehouse}       & 250,000  &           &      \\ \hline
\end{tabular}
\end{table}


The equipment schedule for elevators is irregular and unpredictable. Therefore, this load does not follow the typical plug load schedule for its associated space type. We decided to approximate an elevator's schedule by relating it to the number of people who are entering or exiting the building at each time step---in other words, the derivative of the occupancy schedule of the building. We also made assumptions regarding the number of people per elevator ride (five), the amount of time per ride (calculated from the elevator speed and number of stories), and the amount of inter-floor traffic that is not captured by the change in building occupancy (added a factor of 1.2x). Elevator data and metrics like this are not commonly measured or available, so these assumptions are based primarily on engineering judgment. 

From these calculations and assumptions, we derived an elevator schedule for each building type and each day of the week. An example of the elevator schedule for medium offices is shown in Figure~\ref{fig:medium_office_elevator_schedule} for weekdays, Saturdays, and Sundays. On weekdays, the most elevator traffic occurs at the beginning and end of the day, when people are coming in or leaving work for the day. There is also significant traffic during the lunch hour, as some people choose to leave the building for lunch. At all other times during the workday, the elevator load is approximately 40\% of the total load to account for inter-floor traffic and minimal change in the total building occupancy. For some buildings, the occupancy schedules are reduced on Saturdays and include no occupancy on Sundays, which is reflected in the elevator schedule. 

\begin{figure}[ht!]
    \centering 
    \includegraphics[]{figures/medium_office_elevator.png}
    \caption[Medium office elevator schedule]{Elevator fractional load schedule for medium offices on weekdays, Saturdays, and Sundays.}
    \label{fig:medium_office_elevator_schedule}
\end{figure}

The final aspect of modeling elevators is accounting for lighting and fans inside the elevator. Although these are minimal loads compared to the total elevator energy, they are still modeled in ComStock. The elevator lighting and fans are defined in the model by a total power in watts. These wattage values were calculated based on the number of elevators and the subsystem template in the model, thereby ensuring higher efficiencies for newer lighting and ventilation systems. The elevator lighting and fan schedules are assumed to be at full load at all times when the elevator schedule is greater than 0.

\pagebreak

\subsection{Kitchen Equipment} 
Kitchens are one of the most energy-intensive space types. In ComStock, kitchen space types are modeled in six building types---FullServiceRestaurant, Hospital, LargeHotel, PrimarySchool, QuickServiceRestaurant, and SecondarySchool. In some building types, namely restaurants, the kitchen space type represents a significant proportion of the floor area. In these cases, kitchen loads have a major impact on the total building EUI. In hotels, hospitals, and schools, the kitchen space type only represents a small fraction of the total floor area. Table \ref{tab:kitchen_floor_area} shows the percent of the total floor area represented by the kitchen space type for each building type.

Most commercial kitchens contain a variety of both electric and gas equipment. Electric equipment in kitchens is modeled using an equipment power density value, as other plug loads are modeled. Electric kitchen loads are included in the electric equipment end use in reporting. In the future, electric kitchen equipment may be split out as its own end use. The equipment power densities for kitchens are mostly derived from the DOE prototype models. A breakdown of electric equipment power densities in kitchen space types is shown in Table ~\ref{tab:kitchen_epds}.

Gas equipment is a very prevalent and energy-intensive load in commercial kitchens. Some common pieces of gas kitchen equipment are broilers, fryers, griddles, ovens, ranges, and steamers. ComStock does not currently model each type of kitchen equipment separately, but this may be added in the future. Gas equipment in ComStock is modeled as an energy intensity in BTU per hour per square foot. The gas equipment energy intensity values for kitchens in each building type are shown in Table \ref{tab:kitchen_gas}. 

\begin{table}[h!]
\small
\centering
\caption[Kitchen Space Type Percentage of Total Floor Area]{Kitchen Space Type Percentage of Total Floor Area}
\label{tab:kitchen_floor_area}
\begin{tabular}{|l|l|l|}
\hline
\textbf{Building Type}          & \textbf{Space Type} & \textbf{Percentage of Total Floor Area} \\ \hline
FullServiceRestaurant  & Kitchen    & 27.3\%                         \\ \hline
Hospital               & Kitchen    & 4.1\%                          \\ \hline
LargeHotel             & Kitchen    & 0.9\%                          \\ \hline
PrimarySchool          & Kitchen    & 2.4\%                          \\ \hline
QuickServiceRestaurant & Kitchen    & 50\%                           \\ \hline
SecondarySchool        & Kitchen    & 1.1\%                          \\ \hline
\end{tabular}
\end{table}
\begin{table}[h]
\small
\centering
\caption[Kitchen Electric Equipment Energy Intensity]{Kitchen Electric Equipment Energy Intensity (W/ft\textsuperscript{2})}
\label{tab:kitchen_epds}
\begin{tabular}{|p{1.27in}|p{0.48in}|p{0.5in}|p{0.5in}|p{0.5in}|p{0.5in}|p{0.5in}|p{0.5in}|}
\hline
\textbf{Building Type} & \textbf{Space Type} & \textbf{DOE Ref Pre-1980} & \textbf{DOE Ref 1980--2004} & \textbf{90.1-2004} & \textbf{90.1-2007} & \textbf{90.1-2010} & \textbf{90.1-2013} \\ \hline
FullServiceRestaurant  & Kitchen  & 15.68  & 15.68 & 15.65 & 15.65 & 15.65 & 15.65 \\ \hline
Hospital               & Kitchen  & 7.50   & 7.50  & 7.50  & 7.50  & 7.50  & 7.50  \\ \hline
LargeHotel             & Kitchen  & 47.20  & 47.20 & 47.20 & 47.20 & 47.20 & 47.20 \\ \hline
PrimarySchool          & Kitchen  & 2.30   & 2.30  & 2.25  & 2.25  & 2.25  & 2.25  \\ \hline
QuickServiceRestaurant & Kitchen  & 12.92  & 12.92 & 12.90 & 12.90 & 12.90 & 12.90 \\ \hline
SecondarySchool        & Kitchen  & 4.41   & 4.41  & 3.57  & 3.57  & 3.57  & 3.57  \\ \hline
\end{tabular}
\end{table}
\begin{table}[h]
\small
\centering
\caption[Kitchen Gas Equipment Energy Intensity]{Kitchen Gas Equipment Energy Intensity (Btu/hr*ft\textsuperscript{2})}
\label{tab:kitchen_gas}
\begin{tabular}{|p{1.27in}|p{0.48in}|p{0.5in}|p{0.5in}|p{0.5in}|p{0.5in}|p{0.5in}|p{0.5in}|}
\hline
\textbf{Building Type} & \textbf{Space Type} & \textbf{DOE Ref Pre-1980} & \textbf{DOE Ref 1980--2004} & \textbf{90.1-2004} & \textbf{90.1-2007} & \textbf{90.1-2010} & \textbf{90.1-2013} \\ \hline
{FullServiceRestaurant}  & {Kitchen}    & {378.8}  & {378.8}  & {205.8} & {205.8}   & {205.8} & {205.8} \\ \hline
{Hospital}               & {Kitchen}    & {96.7}   & {96.7}   & {51.2}  & {51.2}    & {51.2}  & {51.2}  \\ \hline
{LargeHotel}             & {Kitchen}    & {511.0}  & {511.0}  & {767.0} & {767.0}   & {767.0} & {767.0} \\ \hline
{PrimarySchool}          & {Kitchen}    & {302.6}  & {302.6}  & {453.7} & {453.7}   & {453.7} & {453.7} \\ \hline
{QuickServiceRestaurant} & {Kitchen}    & {409.6}  & {409.6}  & {250.9} & {250.9}   & {250.9} & {250.9} \\ \hline
{SecondarySchool}        & {Kitchen}    & {354.9}  & {354.9}  & {532.1} & {532.1}   & {532.1} & {532.1} \\ \hline
\end{tabular}
\end{table}

 