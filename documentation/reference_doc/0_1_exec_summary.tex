\chapter{Executive Summary} 
The commercial building sector stock model, or ComStock\textsuperscript{TM}, is a highly granular, bottom-up model that uses multiple data sources, statistical sampling methods, and advanced building energy simulations to estimate the annual subhourly energy consumption of the commercial building stock across the United States. This document contains the methodology and assumptions behind ComStock and serves as a guide to its use.

\paragraph{Motivation: Why Do We Need a Commercial Building Stock Energy Model?}
Across the United States, the number of decarbonization initiatives in cities, counties, and states continues to grow. The goals of these initiatives are often aspirational, targeting 100\% renewable energy by a specific date for a specific geographic area. When considering the task of decarbonizing the energy system, electric grid supply-side generation technologies such as solar photovoltaics (PV) and wind are often the first technologies that come to mind. However, the energy system's demand side also offers significant decarbonization opportunities. In the United States, on-site fossil fuel combustion, primarily for space and water heating, accounts for 40\% of on-site energy usage in commercial buildings \citep{eia2018cbecs}. Even if a grid is converted to 100\% renewable energy, more than half of on-site energy consumption remains to be decarbonized. A major effort is required to achieve clean energy goals on the demand side, and it falls on public sector staff, the engineering and policy consulting communities, and research organizations to ensure that these goals are realistic, equitable, and achievable.

Understanding how the commercial building stock uses energy is a first step toward meeting these goals. The U.S. commercial building stock consumes 11\% of the natural gas and 34\% of the electricity used in the country \citep{eia_natural_gas}. This consumption, plus a smaller amount of other fuels, means that the commercial sector is responsible for 16\% of U.S. CO\textsubscript{2} emissions associated with energy consumption \citep{eia_emissions_from_energy_consumption}.

To make informed decisions about emissions, it is necessary to understand when and where energy consumption is happening in more detail. The U.S. Energy Information Administration's (EIA) Commercial Building Energy Consumption Survey (CBECS) collects detailed information on energy consumption and building characteristics of the commercial sector. However, CBECS data are only provided at an annual timescale and at a census division spatial resolution. As the grid evolves, the timing of energy consumption is becoming more important in decision-making, and policies impacting energy consumption are increasingly being made at the state and city level. Decision makers need a tool to evaluate the impact of potential changes to their building stock, and need the results to be relevant to their local building stock and grid context.

Currently available energy analysis tools including energy audits and building energy models focus on individual buildings and a static power grid mix. However, the power grid mix continues to change as it incorporates wind, solar PV, batteries and electric vehicle charging. Advanced building controls and demand response programs make possible grid-interactive efficient buildings (GEB) that can achieve greater savings by responding to real-time changes in the power grid. Time becomes a vitally important factor when considering the changing energy supply and demand environment. The time of day or night when building energy efficiency measures provide energy savings needs to be identified and then correlated with the power grid mix. Do energy savings occur at night when wind is on the grid or during daytime PV production? These considerations are further impacted by the geographic location and local climate.

To meet clean energy goals and improve integration of the building stock with a changing power grid, a comprehensive analysis technique is required that can simultaneously analyze where, when, and how groups of buildings consume and could save energy. The ComStock analysis tool was developed by the National Renewable Energy Laboratory (NREL) with funding from the U.S. Department of Energy (DOE) to assist the professionals and researchers tasked with implementing these initiatives.